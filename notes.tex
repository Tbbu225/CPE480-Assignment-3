%
% The first command in your LaTeX source must be the \documentclass command.
\documentclass[sigconf]{acmart}

%
% defining the \BibTeX command - from Oren Patashnik's original BibTeX documentation.
\def\BibTeX{{\rm B\kern-.05em{\sc i\kern-.025em b}\kern-.08emT\kern-.1667em\lower.7ex\hbox{E}\kern-.125emX}}  

 % Rights management information. 
% This information is sent to you when you complete the rights form.
% These commands have SAMPLE values in them; it is your responsibility as an author to replace
% the commands and values with those provided to you when you complete the rights form.
%
% These commands are for a PROCEEDINGS abstract or paper.
\copyrightyear{2019}
\acmYear{2019}
\setcopyright{acmlicensed}
\acmConference{N.A}
\acmBooktitle{N.A.}
\acmPrice{0.00}
\acmDOI{N.A}
\acmISBN{N.A.}

% end of the preamble, start of the body of the document source.
\begin{document}

%
% The "title" command has an optional parameter, allowing the author to define a "short title" to be used in page headers.
\title{CPE480 Assignment \#3 }

%
% The "author" command and its associated commands are used to define the authors and their affiliations.
% Of note is the shared affiliation of the first two authors, and the "authornote" and "authornotemark" commands
% used to denote shared contribution to the research.
\author{Tyler Burkett}
\email{tbbu225@uky.edu}
\orcid{1234-5678-9012}
\affiliation{%
  \institution{University of Kentucky}
  \streetaddress{P.O. Box 1212}
  \city{Lexington}
  \state{Kentucky}
  \postcode{40506-0107}
}

\author{Jarren Tay}
\email{jarrentay@uky.edu}
\orcid{1234-5678-9012}
\affiliation{%
  \institution{University of Kentucky}
  \streetaddress{P.O. Box 1212}
  \city{Lexington}
  \state{Kentucky}
  \postcode{40506-0107}
}

\author{Evan Jones}
\email{sejo238@uky.edu}
\orcid{1234-5678-9012}
\affiliation{%
  \institution{University of Kentucky}
  \streetaddress{P.O. Box 1212}
  \city{Lexington}
  \state{Kentucky}
  \postcode{40506-0107}
}
%
% By default, the full list of authors will be used in the page headers. Often, this list is too long, and will overlap
% other information printed in the page headers. This command allows the author to define a more concise list
% of authors' names for this purpose.
\renewcommand{\shortauthors}{Burkett, Jones, Tay}

%
% The abstract is a short summary of the work to be presented in the article.
\begin{abstract}
This project is TACKY, a twin accumulator processor that interprets 16 bit instruction words with up to 2 instructions per instruction word. TACKY is pipelined into 5 stages, so it can process up to two instructions per clock cycle. For the sake of simplicity, this hardware handles dependencies
\end{abstract}

%
% The code below is generated by the tool at http://dl.acm.org/ccs.cfm.
% Please copy and paste the code instead of the example below.
%
  \begin{CCSXML}
<ccs2012>
<concept>
<concept_id>10010520.10010521.10010522.10010526</concept_id>
<concept_desc>Computer systems organization~Pipeline computing</concept_desc>
<concept_significance>500</concept_significance>
</concept>
<concept>
<concept_id>10010520.10010521.10010528.10010529</concept_id>
<concept_desc>Computer systems organization~Very long instruction word</concept_desc>
<concept_significance>300</concept_significance>
</concept>
<concept>
<concept_id>10010520.10010521.10010522.10010523</concept_id>
<concept_desc>Computer systems organization~Reduced instruction set computing</concept_desc>
<concept_significance>100</concept_significance>
</concept>
</ccs2012>
\end{CCSXML}

\ccsdesc[500]{Computer systems organization~Pipeline computing}
\ccsdesc[300]{Computer systems organization~Very long instruction word}
\ccsdesc[100]{Computer systems organization~Reduced instruction set computing}
%
% Keywords. The author(s) should pick words that accurately describe the work being
% presented. Separate the keywords with commas.
\keywords{ Pipeline, VLIW, RISC Instruction Set, TACKY, accumulator-based architecture}

%
% This command processes the author and affiliation and title information and builds
% the first part of the formatted document.
\maketitle

\section{General Approach}

TACKY uses what is called a Very Long Instruction Word (VLIW) that can have one or two instructions in the word. These VLIWs are only 16 bits, so for words that have two instructions, 8 bits are used to define each instruction. 5 bits indicate the operation to perform. 3 bits indicate one of the registers to use. The position of the instruction (whether it appears first or second in the word) determines the second implicit register that is used.

TACKY will also be able to process both integer and floating point 16-bit instructions. To allow for int and float differentiation, our registers are tagged: 0 for integer and 1 for floating point. This means that each register is actually 17 bits. Because our VLIW is only 16 bits, we can only interpret 8-bit constants. To work up to 16 bits, we have an instruction called “pre” that is used to load the first half of a 16-bit constant in. This value is then prepended to the immediates of other instructions that take 8 bits.

To process our instructions, we implemented a five-stage pipeline. The stages are as follows: Instruction Fetch, Register Read, ALU/MEM, ALU 2, Register Writeback. In between each stage, we have a register that takes the output of one stage and temporarily stores it for the next stage. 

\subsection{Stage 0: Instruction Fetch}
In our Instruction Fetch stage, we fetch the instruction from memory, decide whether we need to stall our pipeline, and detect and initiate the halting procedure.

\subsection{Stage 1: Register Read}
In our Register Read stage, we check the instructions being used in the operations we need to perform and fetch the values of the appropriate registers. In addition, we also perform the pre operation.

\subsection{Stage 2: ALU / Data Memory}
In our ALU/MEM stage, we have two ALUs. There are two goals for these ALUs: to perform arithmetic operations on the register values and to load from and store to memory. Each receive the instruction word, and the values of the registers it would be concerned about. If the instruction word had an immediate, it was fed to the first ALU and the second ALU does nothing. Most instructions will finish in this stage, with the exceptions of load float, load int, and floating-point divide. Load float and load int will read the 16-bit value from memory. Floating point divide will read from the reciprocal lookup table. All inputs to this stage are transferred to the next stage. An intermediate "out value" is also transferred to the next stage.

\subsection{Stage 3: ALU2}
In our ALU 2 stage, we finish up the operations that we didn’t finish in the previous stage. Again, there are two ALUs, and we receive the same things as input as the first stage of ALUs. In addition, we receive the intermediate value. This intermediate value is going to be the output for most instructions. For load float, we prepend a 1 to the value we read to signify that it is a float. For load int, we prepend a 0 instead. For floating point divide, we multiply the value from the lookup table by the accumulator. Again, we pass the instruction word, the register values, and the alu generated values to the next stage.

\subsection{Stage 1: Write Back}
In our Writeback stage, we modify our pc (change if jump, don’t change if stalling, increment if otherwise). We write any registers that were modified to our register file.

\subsection{Dependencies and Jump Handling}
Because our processor is pipelined, the effects of a previous instruction may not have taken effect when we begin processing the next one. For example, if two sequential instructions both read and modify the same register, the second instruction would read the old register value before the first instruction would update it. To solve this issue, we need to stall stages of our pipeline.

To check whether we need to stall, we first look at the instruction we are about to process, and we determine which registers are being read for this type of instruction. Then, we look at the instructions being processed at the next three stages in order. If those instructions are modifying a register that we need to read, we'll need to stall. To stall, we output nops for the next X clock cycles and freeze the pc. X is dependent on how which stage we caught the dependency. 3 for Reg Read stage, 2 for ALU/MEM stage, and 1 for Writeback stage. Because we freeze the pc for this time, we will execute the correct instruction after getting through the nops. In addition to register dependencies, we also stall if we detect a jump instruction. For jumps, we stall until the instruction gets to the Writeback stage, so that if we need to jump, it will modify the pc. 

%How to include a picture in this format
%\begin{figure}[h]
% \centering
%  \includegraphics[width=\linewidth]{}
%  \caption{}
%  \Description{}
%\end{figure}

\section{Testing}
Our test cases are included in a separate file. Test cases were written to cover all memory, jump, and integer arithmetic instructions. 

\section{Issues}
During testing, our processor could properly read and parse instructions until we reached a jump instruction followed by a sys instruction. We supplied the first stage with a jump flag and the pc to jump to, but because it was followed by a sys, our processor parsed the sys instead of jumping.

\end{document}
