%
% The first command in your LaTeX source must be the \documentclass command.
\documentclass[sigconf]{acmart}

%
% defining the \BibTeX command - from Oren Patashnik's original BibTeX documentation.
\def\BibTeX{{\rm B\kern-.05em{\sc i\kern-.025em b}\kern-.08emT\kern-.1667em\lower.7ex\hbox{E}\kern-.125emX}}  

 % Rights management information. 
% This information is sent to you when you complete the rights form.
% These commands have SAMPLE values in them; it is your responsibility as an author to replace
% the commands and values with those provided to you when you complete the rights form.
%
% These commands are for a PROCEEDINGS abstract or paper.
\copyrightyear{2019}
\acmYear{2019}
\setcopyright{acmlicensed}
\acmConference{N.A}
\acmBooktitle{N.A.}
\acmPrice{0.00}
\acmDOI{N.A}
\acmISBN{N.A.}

% end of the preamble, start of the body of the document source.
\begin{document}

%
% The "title" command has an optional parameter, allowing the author to define a "short title" to be used in page headers.
\title{CPE480 Assignment \#3 }

%
% The "author" command and its associated commands are used to define the authors and their affiliations.
% Of note is the shared affiliation of the first two authors, and the "authornote" and "authornotemark" commands
% used to denote shared contribution to the research.
\author{Tyler Burkett}
\email{tbbu225@uky.edu}
\orcid{1234-5678-9012}
\affiliation{%
  \institution{University of Kentucky}
  \streetaddress{P.O. Box 1212}
  \city{Lexington}
  \state{Kentucky}
  \postcode{40506-0107}
}

\author{Jarren Tay}
\email{jarrentay@uky.edu}
\orcid{1234-5678-9012}
\affiliation{%
  \institution{University of Kentucky}
  \streetaddress{P.O. Box 1212}
  \city{Lexington}
  \state{Kentucky}
  \postcode{40506-0107}
}

\author{Evan Jones}
\email{sejo238@uky.edu}
\orcid{1234-5678-9012}
\affiliation{%
  \institution{University of Kentucky}
  \streetaddress{P.O. Box 1212}
  \city{Lexington}
  \state{Kentucky}
  \postcode{40506-0107}
}
%
% By default, the full list of authors will be used in the page headers. Often, this list is too long, and will overlap
% other information printed in the page headers. This command allows the author to define a more concise list
% of authors' names for this purpose.
\renewcommand{\shortauthors}{Burkett, Jones, Tay}

%
% The abstract is a short summary of the work to be presented in the article.
\begin{abstract}
\end{abstract}

%
% The code below is generated by the tool at http://dl.acm.org/ccs.cfm.
% Please copy and paste the code instead of the example below.
%
  \begin{CCSXML}
<ccs2012>
<concept>
<concept_id>10010520.10010521.10010522.10010526</concept_id>
<concept_desc>Computer systems organization~Pipeline computing</concept_desc>
<concept_significance>500</concept_significance>
</concept>
<concept>
<concept_id>10010520.10010521.10010528.10010529</concept_id>
<concept_desc>Computer systems organization~Very long instruction word</concept_desc>
<concept_significance>300</concept_significance>
</concept>
<concept>
<concept_id>10010520.10010521.10010522.10010523</concept_id>
<concept_desc>Computer systems organization~Reduced instruction set computing</concept_desc>
<concept_significance>100</concept_significance>
</concept>
</ccs2012>
\end{CCSXML}

\ccsdesc[500]{Computer systems organization~Pipeline computing}
\ccsdesc[300]{Computer systems organization~Very long instruction word}
\ccsdesc[100]{Computer systems organization~Reduced instruction set computing}
%
% Keywords. The author(s) should pick words that accurately describe the work being
% presented. Separate the keywords with commas.
\keywords{ Pipeline, VLIW, RISC Instruction Set, TACKY, accumulator-based architecture}

%
% This command processes the author and affiliation and title information and builds
% the first part of the formatted document.
\maketitle

\section{General Approach}
The general approach to design was that nothing was done. 

%How to include a picture in this format
%\begin{figure}[h]
% \centering
%  \includegraphics[width=\linewidth]{}
%  \caption{}
%  \Description{}
%\end{figure}

\section{Testing}
No testing was done because there was nothing to test.

\section{Issues}
There were no issues because there was nothing to have an issue with.

\end{document}
